\documentclass[]{article}
\usepackage{lmodern}
\usepackage{amssymb,amsmath}
\usepackage{ifxetex,ifluatex}
\usepackage{fixltx2e} % provides \textsubscript
\ifnum 0\ifxetex 1\fi\ifluatex 1\fi=0 % if pdftex
  \usepackage[T1]{fontenc}
  \usepackage[utf8]{inputenc}
\else % if luatex or xelatex
  \ifxetex
    \usepackage{mathspec}
  \else
    \usepackage{fontspec}
  \fi
  \defaultfontfeatures{Ligatures=TeX,Scale=MatchLowercase}
\fi
% use upquote if available, for straight quotes in verbatim environments
\IfFileExists{upquote.sty}{\usepackage{upquote}}{}
% use microtype if available
\IfFileExists{microtype.sty}{%
\usepackage{microtype}
\UseMicrotypeSet[protrusion]{basicmath} % disable protrusion for tt fonts
}{}
\usepackage[margin=1in]{geometry}
\usepackage{hyperref}
\hypersetup{unicode=true,
            pdftitle={PDAD 2018},
            pdfauthor={Thiago Mendes Rosa},
            pdfborder={0 0 0},
            breaklinks=true}
\urlstyle{same}  % don't use monospace font for urls
\usepackage{graphicx,grffile}
\makeatletter
\def\maxwidth{\ifdim\Gin@nat@width>\linewidth\linewidth\else\Gin@nat@width\fi}
\def\maxheight{\ifdim\Gin@nat@height>\textheight\textheight\else\Gin@nat@height\fi}
\makeatother
% Scale images if necessary, so that they will not overflow the page
% margins by default, and it is still possible to overwrite the defaults
% using explicit options in \includegraphics[width, height, ...]{}
\setkeys{Gin}{width=\maxwidth,height=\maxheight,keepaspectratio}
\IfFileExists{parskip.sty}{%
\usepackage{parskip}
}{% else
\setlength{\parindent}{0pt}
\setlength{\parskip}{6pt plus 2pt minus 1pt}
}
\setlength{\emergencystretch}{3em}  % prevent overfull lines
\providecommand{\tightlist}{%
  \setlength{\itemsep}{0pt}\setlength{\parskip}{0pt}}
\setcounter{secnumdepth}{0}
% Redefines (sub)paragraphs to behave more like sections
\ifx\paragraph\undefined\else
\let\oldparagraph\paragraph
\renewcommand{\paragraph}[1]{\oldparagraph{#1}\mbox{}}
\fi
\ifx\subparagraph\undefined\else
\let\oldsubparagraph\subparagraph
\renewcommand{\subparagraph}[1]{\oldsubparagraph{#1}\mbox{}}
\fi

%%% Use protect on footnotes to avoid problems with footnotes in titles
\let\rmarkdownfootnote\footnote%
\def\footnote{\protect\rmarkdownfootnote}

%%% Change title format to be more compact
\usepackage{titling}

% Create subtitle command for use in maketitle
\providecommand{\subtitle}[1]{
  \posttitle{
    \begin{center}\large#1\end{center}
    }
}

\setlength{\droptitle}{-2em}

  \title{PDAD 2018}
    \pretitle{\vspace{\droptitle}\centering\huge}
  \posttitle{\par}
    \author{Thiago Mendes Rosa}
    \preauthor{\centering\large\emph}
  \postauthor{\par}
      \predate{\centering\large\emph}
  \postdate{\par}
    \date{2019-07-30}

\usepackage[brazilian]{babel}
\usepackage[utf8]{inputenc}
\usepackage{float}

\begin{document}
\maketitle

\section{PDAD 2018}\label{pdad-2018}

A PDAD 2018 visitou \textbf{21.908} domicílios e coletou informações de
69.654 moradores.

\subsection{Estrutura etária}\label{estrutura-etaria}

A estrutura etária dos moradores do Distrito Federal é apresentada na
Figura \ref{fig:piramide}.

\begin{figure}[H]
\includegraphics{relatório_pdad_curso_files/figure-latex/piramide-1} \caption{Pirâmide etária \label{fig:piramide}}\label{fig:piramide}
\end{figure}

Os valores específicos podem ser verificados na tabela
\ref{tab:piramide}.

\begin{table}[!hb]
\centering
\begin{tabular}{llrrr}
  \hline
idade\_faixas & sexo & n & n\_low & n\_upp \\ 
  \hline
0 a 4 anos & Feminino & 96.993,00 & 91.892,68 & 102.093,32 \\ 
  0 a 4 anos & Masculino & 102.054,00 & 97.270,90 & 106.837,10 \\ 
  5 a 9 anos & Feminino & 88.139,00 & 82.989,00 & 93.289,00 \\ 
  5 a 9 anos & Masculino & 92.894,00 & 88.128,22 & 97.659,78 \\ 
  10 a 14 anos & Feminino & 103.596,00 & 98.164,54 & 109.027,46 \\ 
  10 a 14 anos & Masculino & 107.671,00 & 102.474,66 & 112.867,34 \\ 
  15 a 19 anos & Feminino & 114.842,00 & 108.576,36 & 121.107,64 \\ 
  15 a 19 anos & Masculino & 117.348,00 & 112.407,03 & 122.288,97 \\ 
  20 a 24 anos & Feminino & 121.383,00 & 115.627,01 & 127.138,99 \\ 
  20 a 24 anos & Masculino & 117.944,00 & 111.778,49 & 124.109,51 \\ 
  25 a 29 anos & Feminino & 126.544,00 & 122.080,60 & 131.007,40 \\ 
  25 a 29 anos & Masculino & 119.316,00 & 114.491,72 & 124.140,28 \\ 
  30 a 34 anos & Feminino & 137.590,00 & 130.546,63 & 144.633,37 \\ 
  30 a 34 anos & Masculino & 126.393,00 & 120.306,54 & 132.479,46 \\ 
  35 a 39 anos & Feminino & 142.423,00 & 136.933,88 & 147.912,12 \\ 
  35 a 39 anos & Masculino & 125.263,00 & 120.020,77 & 130.505,23 \\ 
  40 a 44 anos & Feminino & 125.699,00 & 120.441,46 & 130.956,54 \\ 
  40 a 44 anos & Masculino & 109.759,00 & 105.199,70 & 114.318,30 \\ 
  45 a 49 anos & Feminino & 104.923,00 & 100.715,54 & 109.130,46 \\ 
  45 a 49 anos & Masculino & 92.130,00 & 87.511,28 & 96.748,72 \\ 
  50 a 54 anos & Feminino & 91.483,00 & 87.401,78 & 95.564,22 \\ 
  50 a 54 anos & Masculino & 78.669,00 & 74.989,91 & 82.348,09 \\ 
  55 a 59 anos & Feminino & 74.973,00 & 71.379,55 & 78.566,45 \\ 
  55 a 59 anos & Masculino & 60.808,00 & 57.180,23 & 64.435,77 \\ 
  60 a 64 anos & Feminino & 58.291,00 & 54.818,44 & 61.763,56 \\ 
  60 a 64 anos & Masculino & 45.354,00 & 42.435,73 & 48.272,27 \\ 
  65 a 69 anos & Feminino & 44.177,00 & 41.610,09 & 46.743,91 \\ 
  65 a 69 anos & Masculino & 33.157,00 & 30.833,16 & 35.480,84 \\ 
  70 a 74 anos & Feminino & 30.557,00 & 28.274,71 & 32.839,29 \\ 
  70 a 74 anos & Masculino & 21.855,00 & 20.031,51 & 23.678,49 \\ 
  75 a 79 anos & Feminino & 20.222,00 & 17.938,61 & 22.505,39 \\ 
  75 a 79 anos & Masculino & 13.985,00 & 12.665,26 & 15.304,74 \\ 
  80 a 84 anos & Feminino & 12.318,00 & 10.580,79 & 14.055,21 \\ 
  80 a 84 anos & Masculino & 8.251,00 & 7.055,64 & 9.446,36 \\ 
  Mais de 85 anos & Feminino & 9.915,00 & 8.532,09 & 11.297,91 \\ 
  Mais de 85 anos & Masculino & 4.935,00 & 3.828,68 & 6.041,32 \\ 
   \hline
\end{tabular}
\caption{Pirâmide etária} 
\label{tab:piramide}
\end{table}

\clearpage

\subsection{Salários}\label{salarios}

No que diz respeito aos salários, sua distribuição por faixas de salário
mínimo\footnote{O salário considerado foi de R\$ 954,00} é apresentada
na Figura \ref{fig:salarios}.

\begin{figure}[H]
\includegraphics{relatório_pdad_curso_files/figure-latex/salarios-1} \caption{Salarios por faixa de SM \label{fig:salarios}}\label{fig:salarios}
\end{figure}

Os dados podem ser consultados na Tabela \ref{tab:salarios}.

\begin{table}[!hb]
\centering
\begin{tabular}{lrrr}
  \hline
faixas\_salario & n & n\_low & n\_upp \\ 
  \hline
Até 1 salário & 149.808,34 & 144.194,92 & 155.421,76 \\ 
  Mais de 1 até 2 salários & 307.819,42 & 300.051,56 & 315.587,28 \\ 
  Mais de 2 até 4 salários & 185.752,94 & 178.593,85 & 192.912,02 \\ 
  Mais de 4 até 10 salários & 166.799,10 & 159.332,75 & 174.265,44 \\ 
  Mais de 10 até 20 salários & 56.617,22 & 52.599,90 & 60.634,54 \\ 
  Mais de 20 salários & 15.292,09 & 12.824,61 & 17.759,57 \\ 
   \hline
\end{tabular}
\caption{Salarios por faixa de SM} 
\label{tab:salarios}
\end{table}

\clearpage
\pagebreak

\subsection{Esgotamento}\label{esgotamento}

Por fim, o esgotamento sanitário é apresentado na Figura
\ref{fig:esgotamento}

\begin{figure}[H]
\includegraphics{relatório_pdad_curso_files/figure-latex/esgotamento-1} \caption{Esgotamento sanitário \label{fig:esgotamento}}\label{fig:esgotamento}
\end{figure}

Os números podem ser consultados na Tabela \ref{tab:esgotamento}.

\begin{table}[!hb]
\centering
\begin{tabular}{lrrr}
  \hline
esgotamento\_caesb & n & n\_low & n\_upp \\ 
  \hline
Com Rede Geral (Caesb) & 0,93 & 0,93 & 0,93 \\ 
  Sem Rede Geral (Caesb) & 0,07 & 0,07 & 0,07 \\ 
   \hline
\end{tabular}
\caption{Esgotamento sanitário} 
\label{tab:esgotamento}
\end{table}


\end{document}
